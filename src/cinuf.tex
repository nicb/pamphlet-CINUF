\documentclass{scrartcl}

\title{Culture is not \emph{user--friendly}}
\author{Nicola Bernardini}
\date{~}

\begin{document}

\maketitle

\section{Introduction -- Consensus Societies}
% * Our modern, liquid societies are essentially consensus democracies. Consensus is at the core of society as a marketing mechanism. In order to sell anything, be it a commodity or a political idea, you need consensus.

\section{User--friendliness}
% * One of the key concepts of consensus is *user-friendliness*. Born in the realm of HCI, user-friendliness has been exported to many other domains because it has become an essential concept to gain mass consensus.

\section{Personal Culture}
% * Personal culture is generally acquired through long, complicated and hard processes of study, attention and practice. As such, it can hardly be considered *user-friendly*.
% * On the other hand, personal culture is generally what creates unmeasurable gaps among people - producing the difference between *de facto* and *de jure* individuals.

\section{Culture and Society}
% * This is why culture is not a useful asset in a consensus democracy. In fact, culture creates more problems (critical attention), politically speaking, than solutions.
% * On the other hand, a society without high level culture is bound to decrease its global quality of life.

\section{Conclusion}
% * In order to promote culture we must stop thinking that we can sell it as a product. Culture is not a product and it does not behave like one.

\end{document}
